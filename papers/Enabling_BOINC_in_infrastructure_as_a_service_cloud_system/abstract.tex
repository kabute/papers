\documentclass[journal abbreviation, manuscript]{copernicus}
\usepackage{color,soul}

\begin{document}

\begin{abstract}
Volunteer or Crowd computing is becoming increasingly popular to solve complex research problems, from an increasing diverse range of areas. The majority of these have been built using the Berkeley Open Infrastructure for Network Computing (BOINC) platform, which provides a range of different services to manage all computation aspects of a project. The BOINC system is ideal in those cases where not only does the research community involved need low cost access to massive computing resource but also that there is a significant public interest in the research done. 

We discuss the way in which Cloud services can help BOINC based projects to deliver results in a fast, on demand manner. This is difficult to achieve using volunteers, and at the same time, using scalable cloud resources for short on demand projects can optimize the use of the available resources. We show how this design can be used as an efficient distributed computing platform within the Cloud, and outline new approaches that could open up new possibilities in this field, using climate\textit{prediction}.net as a case study.

\keywords{BOINC, Cloud, CPDN, Volunteer computing}
\end{abstract}
\end{document}
